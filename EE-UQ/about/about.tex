The audience of this tool is researchers and practitioners trying to
predict the response of a structure to earthquake events.\\

This open-source research application, the source code of which is
available at the \href{https://github.com/NHERI-SimCenter/EE-UQ}{\texttt{\getsoftwarename{}}
Github page}, provides an application researchers can use to predict
the response of a building subjected to earthquake events. The
application is focused on quantifying the uncertainties in the
predicted response, given the that the properties of the buildings and
the earthquake events are not known exactly, and that both the
simulation software and the user make simplifying assumptions in the
numerical modeling of that structure. In this application, the user is
required to characterize the uncertainties in the input. The
application will, after utilizing the users selected sampling method,
provide information that characterizes the uncertainties in the
computed response measures. As the computations to make these
determinations can be prohibitively expensive to perform on a users
local computer, the user has the option to perform the computations
remotely on the Stampede2 supercomputer. Stampede2 is located at the
Texas Advanced Computing Center and made available to the user through
NHERI DesignSafe, the cyberinfrastructure provider for the distributed
NSF funded Natural Hazards in Engineering Research Infrastructure
(NHERI) facility.\\

Whether running locally or remotely, the computations are performed,
as will be discussed in \Cref{chap:theory}, in a workflow
application. That is, the numerical simulations are actually performed
by a number of different applications. The \texttt{\getsoftwarename{}} backend software runs
these different applications for the user, taking the outputs from
some programs and providing them as inputs to others. The design of
the \texttt{\getsoftwarename{}} application is such that researchers are able to modify the
backend application to utilize their own application in the workflow
computations. This will ensure researchers are not limited to using
the default applications we provide and will be enthused to provide
their own applications for others to use. \\

This is Version 1.1 of the tool. Researchers are encouraged to comment
on what additiona; features and applications they would like to see in
this application. If you want it, chances are many of your colleagues
also would benefit from it. The user is encouraged to review
Section \Cref{chap:requirements} to see what features are planned for this
application.
