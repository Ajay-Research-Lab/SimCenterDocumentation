This user requirements chapter outlines in general the features provided by the tool, when the features were introduced, and what features we are expecting to introduce in the future. It provides a roadmap of where this tool has come from and where we think it is going. The future features are highly dependent on user feedback. You are highly encouraged to contact us to discuss any new features you would like to see added to the application. \\

The \Cref{tab:schedule} shows the scheduled release dates for this tool and includes the list of features provided in that release or which are expected to be provided in the future release. individual requirements that are expected to be provided in that release. These requirements are listed in \Cref{tab:featureRequirements}.

\begin{table}[hbt!]                    
  \centering
\begin{adjustbox}{max width=\textwidth}            
  \begin{tabular}{lll}                    
    \toprule          
      Version & 	Release	 & Requirements \\  \hline
      1.0	& Sept 2018 &	1.1, 1.2, 2.1, 2.2, 2.3, 2.4, 2.5, 3.1, 4.1, 4.2, 5.1, 5.2, 6.1, 7\\  \hline
      1.1	 & Mar 2019 &	2.7.a, 2.8.a, 3.6, 6.3, 8.1\\  \hline
      1.2	 & Jun 2019 &	2.7.b, 2.8.b, 6.2 \\  \hline
      2.0	 & Sep 2019 &	2.6, 3.2, 4.3, 6.2, 6.3\\  \hline
  \end{tabular}
\end{adjustbox}
  \caption{Schedule of Release}             
  \label{tab:schedule}                 
\end{table}

The following table outlines the user requirements identified for this
tool. If you would like some additional features added, please contact us.
Additionally, we would appreciate any feedback on this tool. An anonymous
user survey is available \insertsurveylink{here}.


\begin{table}[hbt!]                 
  \centering
\begin{adjustbox}{max width=\textwidth}            
  \begin{tabular}{llll}                    
    \toprule          
      \# & Description & Priority & Version \\ \hline
    
      1 & Ability to perform UQ on Building with Single Earthquake &  &  \\ \hline
	1.1 & Run on Local Machine (Mac and Windows) & M & 1.0 \\ \hline
	1.2 & Run on Stampede2 through DesignSafe utilizing Agave & M & 1.0 \\ \hline
	2 & Motion Selection &  &  \\ \hline
	2.1 & Ability to select from Multiple Earthquakes and view UQ due to all the discrete events & M & 1.0  \\ \hline
	2.2 & Ability to select from list of SimCenter motions & M & 1.0 \\ \hline
	2.3 & Ability to select from list of PEER motions. & D & 1.0 \\ \hline
	2.4 & Ability to use OpenSHA and selection methods to generate motions & D & 1.0 \\ \hline
	2.5 & Ability to Utilize Own Application in Workflow & M & 1.0 \\ \hline
	2.6 & Ability to use Broadband & D & 2.0 \\ \hline
	\multirow{5}{*}{2.7} 
	& Ability to bring motion from rock to surface through soil &  &  \\ 
	 & a)     1d soil effective stress analysis though different soil layers & M & 1.1  \\ 
	 & b)     2d bidirectional loading & M & 1.2 \\ 
	 & c)     2d bidirectional with full stochastic characterization of soil layers & M & 1.3 \\ \hline

	\multirow{5}{*}{2.8} 
	& Ability to select from synthetic ground motions &  &  \\ 
	 & a)     per Vlachos, Papakonstantinou, Deodatis (2017) & D & 1.1  \\ 
	 & b)     per Dabaghi, Der Kiureghian (2017) & D & 1.2 \\ \hline
	3 & Building Model Generation &  &  \\ \hline
	3.1 & Ability to use existing OpenSees model scripts. & M & 1.0 \\ \hline
	\multirow{5}{*}{3.2}  & Ability to define building and use Expert System to generate FE mesh. &  &  \\
	 & a)     Concrete Shear Walls & M & 2.0 \\ 
	 & b)     Moment Frames & M & 2.0 \\ 
	 & c)     Braced Frames & M & 2.0  \\ \hline
	 
	\multirow{5}{*}{3.3} & Ability to define building and use Machine Learning applications to generate FE mesh for: &  &  \\ 
	 & d)     Concrete Shear Walls & M & 2.0 \\ 
	 & e)     Moment Frames & M & 2.2 \\ 
	 & f)     Braced Frames & M & 2.3  \\ \hline

	3.4 & Ability to specify connection details for member ends & M & 2.2 \\ \hline
	3.5 & Ability to define a user-defined moment-rotation response representing the connection details & D & 2.2 \\ \hline
	3.6 & Ability to quickly create a simple shear building model & D & 1.1 \\ \hline
	4 & FEM &  &  \\ \hline
	4.1 & Ability to specify OpenSees as FEM engine and to specify different analysis options. & M & 1.0 \\ \hline
	4.2 & Ability to provide own OpenSees Analysis script to OpenSees engine. & D & 1.0 \\ \hline
	4.3 & Ability to use alternative FEM engine. & M & 2.0 \\ \hline
	5 & UQ - Method &  &  \\ \hline
	5.1 & Ability to Use Dakota UQ engine with the Monte Carlo and LHS methods to perform sampling & M & 1.0 \\ \hline
	5.2 & Ability to Use alternative UQ engines & M & 2.0 \\ \hline
	6 & UQ – Random Variables &  &  \\ \hline
	\multirow{5}{*}{6.1} & Ability to Define Variables of certain types: &  &  \\ 
	 & a)     Normal &  &  \\ 
	 & b)     Lognormal &  &  \\ 
	 & c)     Uniform & M  & 1.0 \\ 
	 & d)     Beta &  &  \\ 
	 & e)     Weibull &  &  \\ 
	 & f)     Gumbel &  &  \\ \hline
	6.2 & User defined Distribution & M & 1.2 \\ \hline
	6.3 & Define Correlation Matrix & M & 1.1 \\ \hline
	7 & Tool to allow user to load and save user inputs & M & 1.0 \\ \hline
	8 & Engineering Demand Parameters &  &  \\ \hline
	8.1 & Ability to Process own Output Parameters & M & 1.1  \\ \hline
	8.1 & Add to Standrard Earthquake a variable indicating analysis failure & D & 1.2  \\ \hline
	9 & Misc. &  &  \\ \hline
	9.1 & Simplify run local and run remote by removing workdir locations. move to preferences & D & 1.2  \\ \hline
      \bottomrule                                  
  \end{tabular}
\end{adjustbox}
  \caption{Feature Rquirements (M=Mandatory, D=Desirable, O=Optional, P=Possible Future)}             
  \label{tab:featureRequirements}                 
\end{table}


