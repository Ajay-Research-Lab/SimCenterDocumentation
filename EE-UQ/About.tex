The audience of this tool is researchers and practitioners trying to
predict the response of a structure to earthquakes.

This open-source research application
(\href{https://github.com/NHERI-SimCenter/EE-UQ}{https://github.com/NHERI-SimCenter/EE-UQ})
provides an application researchers can use to predict the response of
a building subjected to earthquake events. The application is focused
on quantifying the uncertainties in the predicted response, given the
that the properties of the buildings and the earthquake events are not
known exactly, and that the simulation software and the user make
simplifying assumptions in the numerical modeling of that
structure. In the application, the user is required to characterize
the uncertainties in the input. The application will after utilizing
the selected sampling method, will provide information that
characterizes the uncertainties in the response measures. The
computations to make these determinations can be prohibitively
expensive. To overcome this impediment the user has the option to
perform the computations on the Stampede2 supercomputer. Stampede2 is
located at the Texas Advanced Computing Center and made available to
the user through NHERI DesignSafe, the cyberinfrastructure provider
for the distributed NSF funded Natural Hazards in Engineering Research
Infrastructure (NHERI) facility.\\

The computations are performed in a workflow application. That is, the
numerical simulations are actually performed by a number of different
applications. The EE-UQ backend software runs these different
applications for the user, taking the outputs from some programs and
providing them as inputs to others. The design of the EE-UQ
application is such that researchers are able to modify the backend
application to utilize their own application in the workflow
computations. This will ensure researchers are not limited to using
the default applications we provide and will be enthused to provide
their own applications for others to use. \\

This is Version 1.0 of the tool and as such is limited in
scope. Researchers are encouraged to comment on what additional
features and applications they would like to see in this
application. If you want it, chances are many of your colleagues also
would benefit from it.\\
\\
