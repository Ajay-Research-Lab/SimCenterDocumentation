This panel is where the user selects the outputs to be displayed when the simulation runs. There are two options available in the pull-down menu:
\begin{enumerate}
\item Standard Earthquake
\item USer Defined
\end{enumerate}

\subsection{Standard Earthquake}
When the user selects standard Earthquake there are no additional inputs required. The standard earthquake EDP generator will ensure the the max absolute value of the following are obtained: \begin{enumerate}
\item Relative Floor displacements:
\item Absolute Floor Accelerations
\item Interstory Drifts
\end{enumerate}

The  results will contain results for these in abbreviated form:
\begin{itemize}
\item PFD peak relative floor displacement $1-PFD-FLOOR_CLINE$
\item PFA peak floor acceleration (relative + ground motion): $1-PFA-FLOOR-CLINE$
\item PID peak inter-story drift: $1-PID-STORY-CLINE$
\end{itemize}

\subsection{User Defined}
This panel allows the user to provde to determine their own output and process it. When using this option the user provides additional data:
\begin{enumerate}
\item Additional Input: These are additional commands that are invoked by the analysis application before the transient analysis is performed. For example, foe OpenSees this would be a script containing a series of recorder commands.
\item Postprocess Script: This is a python script that will be invoked after the finite element application has run. It must be provided by the user. It's purpose is to process the output files and create a single file, results.out. This file must contain a single line with as many entries as EDP's specified.

\item Response Parameters. This is an area in which the user associates a variable name with the column of the results output file. If the process script has an array of strings named named EDP's the script, the Response Parameters will be initially set with these values from the script.
\end{enumerate}

