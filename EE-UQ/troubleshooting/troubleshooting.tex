The EE-UQ can be a complicated tool and it will not always run. Causes of failure include incorrect set up, non-functioing or poorly functioining websites, and of course user error. To discover the errors it is useful to understand how the UI and the backend work when the user submits to run a job. A number of things occur when the Submit button is clicked: 

\begin{enumerate}
\item The UI creates a folder in the workging dir location specified called tmp.SimCenter and in that folder creates another folder called templatedir.
\item The UI then iterates through all the widgets chosen and these widgets place all needed files for the computation into the templatedir directory.
\item A python script is run in this templatedir directory that creates the input file for the UQ Engine. For example, using Dakota the input file dakota.in is created and placed in tmp.SimCenter folder.
\item The UQ engine is then started and runs using the dakota.in input file.
\item As the UQ engine runs, it creates folders in tmp.SimCenter, one folder for each deterministic run.
\item When completed the UQ engine leaves the results files in the tmp.SimCenter folder.
\item The results files are then processed by the UI and presented to the user in the RES tab.
\end{enumerate}

The following is a list of things that we have observed to go wrong when the UI informs the user of a failure and steps the user can take to fix the problem:  

\begin{enumerate}
\item \textbf{Could not create working dir}: The user does not have permission to create the tmp.SimCenter folder in working dir location. Change the Working Dir location in the window that pops up.
\item \textbf{No Script File}: The user has changed the Applications dir location, or the applications folder that accompanies the installation has been modified. Either set the correct dir location or re-install the tool.
\item \textbf{ERROR: Dakota failed to finish}: This can occur for a number of reasons. Go to the tmp.SimCenter folder and have a look for the dakota.err file.
\begin{enumerate}
\item \textbf{No dakota.err file and no dakota.in file}: the python script in templatdir failed to create the necessary files. Have a look at the workflow log file in templatedir folder to see what the error is as it could indicate an error in your input.
\item \textbf{No dakota.err and dakota.in exists}: Dakota failed to run. Check install of Dakota.
\item \textbf{dakota.err file exists}: Open the file and see what the error is.  For example if it says \textbf{Error: at least one variable must be specified.} This means no random variables have been specified. So whether you have only one  deterministic event or you have not specified any random variables in the EDP.
\item \textbf{dakota.err file exists but is empty}: This means that Dakota ran but there was a problem with the simulation. Go to one of the workdir locations. There is a file workflow driver that can be run. Run it and see what the errors are.
\item \textbf{You ran at DesignSafe and no dakota.out files come back}: Go to your data depot older at DesignSafe using the browser. Go to archive/jobs and use the job number shown in table that pops up when you ask to get the job from DesignSafe. look at the .err file in that directory for a clues to as what went wrong.
\item \textbf{No results and you used the Site Response to create the event}. You must run a simulated event in the Site Response Widget before you can submit a job to run.
\end{enumerate}
\end{enumerate}

If still having trouble, you can always join the EE-UQ slack channel and look for similar issues or post a new one.

