The intended audience for the \texttt{\getsoftwarename{}} Application (\texttt{\getsoftwarename{}} App) is researchers and practitioners
interested in performing site-specific analysis of soil in  response to earthquakes. \texttt{\getsoftwarename{}}  is the acronym of site-specific seismic hazard analysis and research kit.  \\

This is an open-source research application. The source code at
the \href{https://github.com/NHERI-SimCenter/s3hark}{\texttt{\getsoftwarename{}}
Github page} provides an application that can be used to analyze the response of soil to earthquake scenarios.
The application focuses on simulating wave propagation along soil depth using finite element (FE) method. 

Given that the properties of the soil layers and the earthquake events are known,
 \texttt{\getsoftwarename{}} provides multiple nonlinear material models for simulating the soil behavior under earthquake loading.



This is Version \getsoftwareversion{} of the tool. Users are
encouraged to comment on what additional features and capabilities
they would like to see in this application. These requests and
feedback can be submitted through an anonymous \insertsurveylink{user
survey}; we greatly appreciate any input you have. If there are
features you want, chances are many of your colleagues also would
benefit from them. 
