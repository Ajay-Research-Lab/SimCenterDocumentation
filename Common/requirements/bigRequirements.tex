

\newcolumntype{C}[1]{>{\centering\arraybackslash}p{#1}}

\begin{longtable}{|  C{.07\textwidth} | p{.5\textwidth} | C{.08\textwidth} | C{.09\textwidth} | C{.08\textwidth} | C{.08\textwidth} |}

\caption{Requirements for Regional Simulations aiding Community Resilience}
\label{tab:regionalRequirements} 
 \\
   \hline
\rowcolor{lightgray}

\textbf{\#} & \textbf{Description} & \textbf{SRC} & \textbf{WBS} & \textbf{PRI} & \textbf{VER} \\ \hline
R1 & Ability to perform regional simulation allowing communities to evaluate resilience and perform what-if types of analysis for natural hazard events & GC  & 1.3.10 rWhale & M & 1.0 \\  \hline
R1.1 & Perform such simulations for ground shaking due to Earthquake & GC & 1.3.10 rWhale & M & 1.0  \\ \hline
R1.2 & Ability to perform such simulations for wave action due to Earthquake induced Tsunami  & GC &  1.3.10 rWhale &  M &   \\  \hline
R1.3 & Ability to perform such simulations for wind action due to Hurricane & GC & 1.3.10 rWhale & M & \\  \hline
R1.4 & Ability to perform such simulations for wave action due to Hurricane Storm Surge & GC & R1.3.10 rWhale & M & \\ \hline
R1.5 & Ability to perform such for multi-hazard simulations: wind + storm surge, rain, wind and water borne debris & GC & 1.3.10 rWhale & M & \\ \hline
R1.6 & Ability to incorporate damage to lifelines in determination of community resilience & GC & 1.3.10 rWhale & M &  \\ \hhline{======}


R2.1 & Ability of stakeholders to perform simulations of different scenarios that aid in planning and response after damaging events & GC & 1.3.10 rWhale & M & 1.0 \\ \hline
R2.2 &  Ability to utilize HPC resources in regional simulations that enables repeated simulation for stochastic modeling & GC & 1.3.10 rWhale & M & 1.0 \\ \hline
R2.3 &  Provide open-source software for developers to test new data and algorithms & GC & 1.3.10 rWhale & M & 1.0  \\ \hline
R2.4 & Ability to use a tool created by linking heterogeneous array of simulation tools to provide a toolset for regional simulation & GC & 1.3.10 rWhale & M & 1.0 \\ \hline
R2.5 &  Ability to utilize existing open-source software for faster deployment & GC & 1.3.10 RDT & M & 1.0 \\ \hline
R2.6 &  Ability to utilize ensemble techniques  & GC & 1.3.10 rWhale & M & 1.0 \\ \hline
R2.7  & Ability to include multi-scale nonlinear models & GC & 1.3.10 rWhale & M & 1.0 \\ \hline
R2.8 & Ability to include a formal treatment of uncertainty and randomness & GC & 1.3.10 rWhale & M & 1.0 \\ \hline
R2.9 & Ability to include latest information and algorithms (i.e. new attenuation models, building fragility curves, demographics, lifeline performance models, network interdependencies, indirect economic loss)
& GC & rWhale 1.3.10 & D & \\ \hline
R2.10 &  Ability to use GIS so communities can visualize hazard impacts & GC & 1.3.10 rWhale & M & \\ \hline
R2.11 &  Ability to explore different strategies in community development, pre-event, early response, and post event, through long term recovery & GC & 1.3.10 RDT & P & \\ \hline
R2.12 &  Ability to use system that creates and monitors real-time data, updats models, incorporates crowdsourcing technologies, and informs decision makers & GC & 1.3.10 RDT & P & \\ \hline
R2.13 &  Ability to use sensor data to update models for simulation and incorporate sensor data into simulation & GC & 1.3.10 RDT & P & \\  \hhline{======}



R3.1& Ability to use open-source version of Hazus & GC & 1.3.6 pelicun & M & 1.0 \\ \hline
R3.2 &  Ability to incorporate improved damage and fragility models for buildings and lifelines & GC & 1.3.6 pelicun & M & 1.0 \\ \hline
R3.3 &  Ability to incorporate improved indirect economic loss estimation models & GC & 1.3.6 pelicun & M & \\ \hline
R3.4 & Ability to include demand surge in determination of damage and loss estimation & GC & 1.3.6 pelicun & M & \\ \hline
R3.5 & Ability to include lifeline disruptions & GC & 1.3.6 pelicun & M & \\ \hhline{======}


R4.1 & Promote 'living' community risk models utilizing local inventory data from various scources & GC & 1.3.0 rWhale & M & \\ \hline
R4.2 & Ability to use cumulative knowledge bases rather than the piecemeal individual approaches & GC & 1.3.3 & M & \\ \hline
R4.3 & Developing and sharing standardized definitions, measurement protocols and strategies for data collection & GC & 1.3.3 & M & \\ \hline
R4.3 & Developing tools that utilize GIS information and online images, e.g. google maps, for data collection & GC & 1.3.4 BRAILS & M & 1.0 \\ \hline
R4.3.1 & Predicting if building is a soft-story building for earthquake simulations & UF & 1.3.4 BRAILS & M & 1.0 \\ \hline
R4.3.2 & Predicting roof shape of building for hurricane wind simulation & SP & 1.3.4 BRAILS & M & 1.0 \\ \hline
R4.3.3 & Predicting level first floor of occupancy for hurricane storm surge simulation & SP & 1.3.4 BRAILS & M &  \\ \hline
R4.4 & Providing instruction on gathering information from WWW for purposes of these regional simulations & UF & 1.2.3 Summer Bootcamp & M & V2.0 \\ \hline
R4.5 & Developing, sharing, and archiving datasets for analyzing and modeling resilience and vulnerability over time & GC & 1.3.3 & M & \\  \hline
R4.6 & Ability to use GIS, high resolution elevation and soil data for wind and storm surge simulations & GC & 1.3.10 & D &  \\ \hline 
R4.7 & Ability to use validated multi-scale models (materials, components, elements) of built environment & GC & 1.3.3 & D & \\ \hline 
R4.8 & Ability to use a national database of models for hazard, buildings, and lifelines created for multiple hazards & GC & P & 1.3.3 & \\ \hhline{======}



5.1  &  Ability to perform validation studies to calibrate accuracy of models & GC & 1.3.10 rWhale & M &    \\ \hline
5 & Identify knowledge gaps and promote NSF generated knowledge through regional demonstration projects that help generate linkages to operational entities and decision makers & GC & 1.4.2 & M &  \\ \hline
5.1 & Provide Earthquake Testbeds& SP & 1.4.2 Testbeds & M & 1.0 \\ \hline
5.1.1 & Provide Bay Area Earthquake Testbed & SP & 1.4.2 Testbeds & M & 1.0 \\ \hline
5.1.2 & Provide Anchorage Earthquake Testbed & SP & 1.4.2 Testbeds & M & 1.1 \\ \hline5
.2 & Provide Tsunami Testbed & SP & 1.4.2 Testbeds & M & \\ \hline
5.3 & Provide Atlantic City Hurricane Wind Testbed & SP & 1.4.2 Testbeds & M &  \\ \hline
5.4 & Provide Atlantic City Hurricane Wind and Storm Surge Testbed & SP & 1.4.2 Testbeds & M & \\ \hline
5.5 & Provide Earthquake and Lifelines Testbed & SP & 1.4.2 Testbeds & M & \\ \hline

\end{longtable}

\noindent
KEY:\\
Source (SRC): GC=Needed for Grand Challenges, SP=Senior Personnel, UF=User Feedback \\
Work Breakdown Structure (WBS): SimCenter WBS  Number \\
Priority (PRI): M=Mandatory, D=Desirable, P=Possible Future \\
Version (VER): Version number the basic requirement was first met 
