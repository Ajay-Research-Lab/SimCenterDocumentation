\begin{longtable}{| p{.07\textwidth} | p{.65\textwidth} | p{.08\textwidth} | p{.08\textwidth} |  p{.08\textwidth} |}

\caption{Requirements for PBE}
  \label{tab:featureRequirements}   
    \\
   \hline
\rowcolor{lightgray}
      \# & Description & Source & Priority & Version \\ \hline
      
      P1 & \textbf{Ability to determine damage and loss calculations for a building subjected to a natural hazard including formal treatment of randomness and uncertainty uncertainty} & GC & M  & 1.0  \\ \hline
      P1.1 & Ability to determine  damage and loss for multiple different hazards & GC & M &  \\ \hline
      P1.1.1 & Damage and Loss for ground shaking due to Earthquake & GC & M & 1.0 \\ \hline
      P1.1.2 & Damage and Loss due to Wind Loading & GC & M &  \\ \hline
      P1.1.3 & Damage and Loss due to water damage due to Tsunami or Coastal Inundation & GC & M &  \\ \hline
     P1.1 & Ability of Practicing Engineers to use multiple coupled resources (applications, databases, viz tools) in engineering practice & GC & 1.0 \\ \hline
P1.2 & Ability to utilize resources beyond the desktop including HPC & GC & M & 1.0 \\ \hline
P1.3 & Tool should incorporate data from WWW & GC & M & 1.0 \\ \hline
P1.4 & Tool available for download from web & GC & M & 1.0 \\ \hline
P1.5 & Ability to use new viz tools for viewing large datasets generated by PBE & GC & M & 1.0 \\ \hhline{=====}
P2 & \textbf{Various Motion Selection Options for Hazard Event} & SP & M & 1.0  \\ \hline

P2.1 & Various Earthquake Events & SP & M & 1.0  \\ \hline
\softwareSwitch{PBE}{ 
 P2.1.1 & Ability to select from Multiple input motions and view UQ due to all the discrete events & GC & M & 1.0  \\ \hline
 P2.1.2 & Ability to select from list of SimCenter motions & SP & M & 1.0 \\ \hline
 P2.1.3 & Ability to select from list of PEER motions & SP & D & 1.0 \\ \hline
 P2.1.4 & Ability to use OpenSHA and selection methods to generate motions & UF & D & 1.0 \\ \hline
 P2.1.5 & Ability to Utilize Own Application in Workflow & SP & M & 1.0 \\ \hline
 P2.1.6 & Ability to use Broadband & SP & D &  \\ \hline
 P2.1.7  & Ability to include Soil Structure Interaction Effects & GC & M & 1.1 \\  \hline
 P2.1.7.1  & 1D nonlinear site response with effective stress analysis & SP & M & 1.1  \\ \hline
 P2.1.7.2  & Nonlinear site response with bidirectional loading & SP & M & 1.2 \\  \hline
 P2.1.7.3  & Nonlinear site response with full stochastic characterization of soil layers & SP & M &  \\ \hline
 P2.1.7.4 & Nonlinear site response, bidirectional different input motions  & SP & M &  \\  \hline
 P2.1.7.5 & Building in nonlinear soil domain utilizing large scale rupture simulation & GC  & M &  \\  \hline
      P2.1.7.5.1 & Interface using DRM method  & SP  & M &  \\  \hline
      P2.1.8 & Utilize PEER NGA www ground motion selection tool  & UF & D & 2.0 \\ \hline
      P2.1.9 & Ability to select from synthetic ground motions & SP & M & 1.0  \\
      P2.1.9.1 & per Vlachos, Papakonstantinou, Deodatis (2017) & SP & D & 1.1  \\ 
      P2.1.9.2 & per Dabaghi, Der Kiureghian (2017) & UF & D & 2.0 \\ \hline
}{
 P2.1  & Ability to select from all EE-UQ Event Options listed in EE-UQ E2& SP & M & 1.0  \\ \hline 
}   


 P2.2 & Various Wind Loading Options & SP & M &   \\ \hline
 \softwareSwitch{PBE}{ 
 } {
 P2.1  & Ability to select from all WE-UQ Event Options listed in WE-UQ W2 & SP & M & 1.0  \\ \hline   
 }
 
 \softwareSwitch{PBE}{ 
 P2.3 & Various Water Loading Options & SP & M &   \\ \hhline{=====}
 } {
 P2.3 & Various Water Loading Options & SP & M &   \\ \hline
 P2.3.1 & Ability to select from all HydroUQ Event Options & SP & M & 1.0  \\ \hhline{=====}
 }
 
 
 P3 & \textbf{Building Model Generation} & GC & M &  \\ \hline
  \softwareSwitch{PBE}{ 
  P3.1 & Ability to quickly create a simple nonlinear building model & GC & D & 1.1 \\ \hline
  P3.2 & Ability to use existing OpenSees model scripts & SP & M & 1.0 \\ \hline
  P3.3  & Ability to define building and use Expert System to generate FE mesh & SP & &  \\ \hline
  P3.3.1 & Expert system for Concrete Shear Walls & SP & M &  \\ \hline
  P3.3.2 & Expert system for Moment Frames & SP & M &  \\ \hline
  P3.3.3 & Expert system for  Braced Frames & SP & M &   \\ \hline
  P3.4 & Ability to define building and use Machine Learning applications to generate FE & GC &  &  \\ \hline
  P3.4.1 & Machine Learning for Concrete Shear Walls & SP & M &  \\ \hline
  P3.4.2 & Machine Learning for Moment Frames & SP & M &  \\ \hline
  P3.4.3 & Machine Learning for Braced Frames & SP & M &   \\ \hline
  P3.5 & Ability to specify connection details for member ends & UF & M &  \\ \hline
  P3.6 & Ability to define a user-defined moment-rotation response representing the connection details & UF & D & 2.2 \\ \hhline{=====}
  } {  
 P3.1 & Ability to Select All Building Model Generators in EE-UQ & SP & M & 1.0 \\ \hhline
 P3.2 & Ability to Select All Building Model Generators in WE-UQ & SP & M &  \\ \hhline
 P3.3 & Ability to Select All Building Model Generators in HydroUQ & SP & M &  \\ \hhline{=====}
  }
 

  P4 & \textbf{Perform Nonlinear Analysis} & GC & M & 1.0 \\ \hline
  P4.1 & Ability to specify OpenSees as FEM engine and to specify different analysis options & SP & M & 1.0 \\ \hline
  P4.2 & Ability to provide own OpenSees Analysis script to OpenSees engine. & SP & D & 1.0 \\ \hline
  P4.3 & Ability to provide own Python script and use OpenSeesPy engine. & UF & O &  \\ \hline
  P4.4 & Ability to use alternative FEM engine. & SP & M & 2.0 \\ \hhline{=====}
  
   P5 & \textbf{Uncertainty Quantification Methods} &  GC & M & 1.0  \\ \hline
   \softwareSwitch{PBE}{ 
  P5.1 & \textbf{Various Forward Propogation Methods} & SP & M & 1.0  \\ \hline
  P5.1.1 & Ability to use basic  Monte Carlo and LHS methods & SP & M & 1.0 \\ \hline
  P5.1.2 & Ability to use Importance Sampling  & SP & M & 2.0 \\ \hline
  P5.1.3 & Ability to use Gaussian Process Regression & SP & M & 2.0 \\ \hline
  P5.1.4 & Ability to use Own External UQ Engine & SP & M &  \\ \hline
  P5.2 & \textbf{Various Reliability Methods} & UF & M &  \\ \hline
  P5.2.1 & Ability to use First Order Reliability method & UF & M &  \\ \hline
  P5.2.2 & Ability to use Second Order Reliability method & UF & M & \\ \hline
  P5.2.2 & Ability to use Surrogate Based Reliability & UF & M & \\ \hline
  P5.2.3 & Ability to use Own External Application to generate Results & UF & M &  \\ \hline
  P5.3 & \textbf{Various Sensitivity Methods} & UF & M &  \\ \hline
  P5.3.1 & Ability to obtain Global Sensitivity Sobol's indices & UF & M &  \\ \hline
  } {
  P5.1 & Ability to use all forward propogation methods available in EE-UQ and WE-UQ  section U1 & SP & M & 1.0 \\ \hhline{=====}
  }
  
   P6 & \textbf{Random Variables for Uncertainty Quantification} & GC & M & 1.0  \\ \hline
   
    \softwareSwitch{PBE}{ 
   P6.1 & Ability to Define Variables of certain types: & SP & M & 1.0  \\ 
   P6.1.1 & Normal & SP & M  & 1.0 \\ \hline
    P6.1.2 & Lognormal & SP & M & 1.0 \\ \hline
    P6.1.3 & Uniform & SP & M & 1.0  \\ \hline
    P6.1.4 & Beta & SP & M & 1.0 \\ \hline
    P6.1.5 & Weibull &  SP & M  & 1.0 \\ \hline
    P6.1.6 & Gumbel &  SP & M & 1.0  \\ \hline
    P6.2 & User defined Distribution & SP & M &  \\ \hline
    P6.3 & Correlated Random Variables & SP & M &  \\ \hline
    P6.4 & Random Fields & SP & M &  \\ \hhline{=====}
    } {
    P6.1 & Ability to use all random variable distributions in EE-UQ and WE-UQ  section U4 & SP & M & 1.0 \\ \hhline{=====}
    }
    
    
   P8 & \textbf{Engineering Demand Parameters} &  &  & \\ \hline
    P8.1 & Ability to Process own Output Parameters & UF & M &   \\ \hline
    P8.2 & Add to Standard Earthquake a variable indicating analysis failure & UF & D &   \\ \hline
    P8.3 & Allow users to provide their own set of EDPs for the analysis. & UF & D & 2.0\\ \hline

P9 & \textbf{Damage and Loss Assessment} & GC & M & 1.0 \\ \hline
P9.1 & Different Assessment Methods & GC & M & 2.0 \\ \hline
P9.1.1 & Ability to perform component-based (FEMA P58) loss assessment for an earthquake hazard. & SP & M & 1.0 \\ \hline
P9.1.2 & Ability to perform component-assembly-based (HAZUS MH) loss assessment for an earthquake hazard. & SP & D & 1.1 \\ \hline
P9.1.3 & Ability to perform downtime estimation using the REDi methodology. & UF & D & \\ \hline
P9.1.4 & Ability to describe building performance with additional decision variables from HAZUS (e.g., business interruption, debris) & SP & D &  \\ \hline
P9.1.5 &  Ability to perform time-based assessment & GC & M &  \\ \hline
P9.1.6 & Ability to perform damage and loss assessment for hurricane wind & GC & M &  \\ \hline
P9.1.7 & Ability to perform damage and loss assessment for storm surge & GC & M &  \\ \hline
P9.2 & Control & SP & M & 1.0 \\ \hline
P9.2.1 & Allow users to set the number of realizations & SP & M & 1.0\\ \hline
P9.2.2 &  Allow users to specify the added uncertainty to EDPs & SP & M & 1.0 \\ \hline
P9.2.3 &  Allow users to decide which decision variables to calculate & SP & D & 1.0 \\ \hline
P9.2.4 &  Allow users to set the number of inhabitants on each floor and customize their temporal distribution. & SP & D & 1.0 \\ \hline
P9.2.5 &  Allow users to specify the boundary conditions of repairability. & SP & D & 1.0 \\ \hline
P9.2.6 &  Allow users to control collapse through EDP limits. & SP & D & 1.0\\ \hline
P9.2.7 &  Allow users to specify the replacement cost and time for the building. & SP & M & 1.0 \\ \hline
P9.2.8 &  Allow users to specify EDP boundaries that correspond to reliable simulation results. & SP & D & 1.0\\ \hline
P9.2.9 & Allow users to specify collapse modes and characterize the corresponding likelihood of injuries. & SP & D & 1.0\\ \hline
P9.2.10 & Allow users to specify the collapse probability of the structure. & UF & M & 1.2\\ \hline
P9.2.11 & Allow users to use empirical EDP data to estimate the collapse probability of the structure. & UF & M & 1.2\\ \hline
P9.2.12 & Allow users to choose the type of distribution they want to estimate the EDPs with. & UF & D & 1.2\\ \hline
P9.2.13 & Allow users to perform the EDP fitting only for non-collapsed cases. & UF & M & 1.2\\ \hline
P9.2.14 & Allow users to couple response estimation with loss assessment. & UF & M & \\ \hline
P9.3 & Component damage and loss information & SP & M & 1.0\\ \hline
P9.3.1 & Make the component damage and loss data from FEMA P58 available. & SP & M & 1.0 \\ \hline
P9.3.2 & Ability to use custom components for loss assessment. & SP & D & 1.0 \\ \hline
P9.3.3 & Allow users to set different component quantities for each floor in each direction. & SP & D & 1.0 \\ \hline
P9.3.4 & Allow users to set the number of identical component groups and their quantities within each performance group. & UF & D & 1.0 \\ \hline
P9.3.5 & Use a generic JSON data format for building components that can be shared by component-based and component-assembly-based assessments. & SP & D & 1.1 \\ \hline
P9.3.6 & Convert FEMA P58 and HAZUS component damage and loss data to the new JSON format and make it available with the tool. & SP & D & 1.1 \\ \hline
P9.3.7 & Make component definition easier by providing a list of available components in the given framework (e.g. FEMA P58 or HAZUS) and not requesting inputs that are already available in the data files. & UF & D & 1.2 \\ \hline
P9.3.8 & Make the component damage and loss data from FEMA P58 2nd edition available. & UF & M & 2.0 \\ \hline
P9.3.9 & Improve component definition by providing complete control over every characteristic on every floor and in every direction & UF & D & 2.0 \\ \hline
P9.3.10 & Allow users to view fragility and consequence functions in the application & UF & D &  \\ \hline
P9.3.11 & Allow users to edit fragility and consequence functions in the application & UF & D &  \\ \hline
P9.4 & Stochastic loss model & SP & M & 1.0 \\ \hline
P9.4.1 & Allow the user to specify basic dependencies (i.e. independence or perfect correlation) between logically similar parts of the stochastic model (i.e. within component quantities or one type of decision variable, but not between quantities and fragilities) & SP & D & 1.0 \\ \hline
P9.4.2 & Allow the user to specify basic dependencies between reconstruction cost and reconstruction time. & SP & D & 1.0 \\ \hline
P9.4.3 & Allow the user to specify basic dependencies between different levels of injuries. & SP & D & 1.0 \\ \hline
P9.4.4 & Allow the user to specify intermediate levels of correlation (i.e. not limited to 0 or 1) and provide a convenient interface that makes sure the specified correlation structure is valid. & SP & D & \\ \hline   
P9.4.5 & Allow the user to specify the correlation for EDPs. & SP & D &  \\ \hhline{=====}

 PM & \textbf{Misc.} & UF & M & 1.2  \\ \hline
    PM.1 & Tool to allow user to load and save user inputs & SP & M & 1.0 \\ \hline
   PM.2 & Simplify run local and run remote by removing workdir locations. Move to preferences & UF & D & 1.2  \\ \hline
  P M.3 & Add to EDP a variable indicating analysis failure & UF & D &   \\ \hline
   PM.4 & Installer which installs application and all needed software & UF & M &   \\ \hhline{=====}
 PE & \textbf{Ability to gain educational materials that will help and encourage PBE} & GC  & M & 1.0 \\ \hline
 PE.1 & Documentation exists on tool usage & SP & M & 1.1  \\ \hline
 PE.2 & Video Exists demonstrating usage & SP & M & 1.1  \\ \hline
 PE.3 & Verification Examples Exist & SP & M & 1.1  \\ \hline
 PE.4 & Validation Examples Exist, validated against tests or other software & GC & D &  \\ \hline
  \bottomrule               
\end{longtable}


\noindent
KEY:\\
Source: GC=Needed for Grand Challenges, SP=Senior Personnel, UF=User Feedback \\
Need: M=Mandatory, D=Desirable, P=Possible Future \\
Version: Version number the basic requirement was met 

