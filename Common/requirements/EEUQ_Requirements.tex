\begin{longtable}{| p{.05\textwidth} | p{.75\textwidth} | p{.08\textwidth} | p{.08\textwidth} |  p{08\textwidth} |}
    \toprule
      \# & Description & Source & Priority & Version \\ \hline
      1 & \textbf{Ability to determine response of Building Subject to Earthquake hazard including formal treatment of randomness and uncertainty uncertainty} & GC & M  & 1.0  \\ \hline
      1.1 & Simulations able to utilize HPC resources & GC & M & 1.0 \\ \hline
      1.2 & Tool should incorporate data from www & GC & M & 1.0 \\ \hline
      1.3 & Tool available for download from web & GC & M & 1.0 \\ \hline
      2 & \textbf{Motion Selection} &  &  \\ \hline
      2.1 & Ability to select from Multiple input motions and view UQ due to all the discrete events & GC & M & 1.0  \\ \hline
      2.2 & Ability to select from list of SimCenter motions & SP & M & 1.0 \\ \hline
      2.3 & Ability to select from list of PEER motions & SP & D & 1.0 \\ \hline
      2.4 & Ability to use OpenSHA and selection methods to generate motions & UF & D & 1.0 \\ \hline
      2.5 & Ability to Utilize Own Application in Workflow & SP & M & 1.0 \\ \hline
      2.6 & Ability to use Broadband & SP & D &  \\ \hline
      2.7  & Ability to include Soil Structure Interaction Effects & GC & M & 1.1 \\  \hline
      2.7.1  & 1D nonlinear site response with effective stress analysis & SP & M & 1.1  \\ \hline
      2.7.2  & nonlinear site response with bidirectional loading & SP & M & 1.2 \\  \hline
      2.7.3  & nonlinear site response with full stochastic characterization of soil layers & SP & M &  \\ \hline
      2.7.4 & nonlinear site response, bidirectional different input motions  & SP & M &  \\  \hline
      2.7.5 & building in nonlinear soil domain - DRM methos & GC  & M &  \\  \hline
      2.8 & Utilize PEER NGA www ground motion selection tool  & UF & D & 2.0 \\ \hline
      2.9 & Ability to select from synthetic ground motions & SP & M & 1.0  \\
      2.9.1 & per Vlachos, Papakonstantinou, Deodatis (2017) & SP & D & 1.1  \\ 
      2.9.2 & per Dabaghi, Der Kiureghian (2017) & UF & D & 2.0 \\ \hline
	3 & \textbf{Building Model Generation} & GC & 2.0 \\ \hline
	3.1 & Ability to use existing OpenSees model scripts & SP & M & 1.0 \\ \hline
	3.2  & Ability to define building and use Expert System to generate FE mesh & SP & &  \\ \hline
	3.2.1 & Expert system for Concrete Shear Walls & SP & M &  \\ \hline
	3.2.2 & Expert system for Moment Frames & SP & M &  \\ \hline
	3.2.3 & Expert system for  Braced Frames & SP & M &   \\ \hline
	3.3 & Ability to define building and use Machine Learning applications to generate FE & GC &  &  \\ \hline
	3.3.1 & Machine Learning for Concrete Shear Walls & SP & M &  \\ \hline
	3.3.2 & Machine Learning for Moment Frames & SP & M &  \\ \hline
	3.3.3 & Machine Learning for Braced Frames & SP & M &   \\ \hline
	3.4 & Ability to specify connection details for member ends & SP & M & 2.2 \\ \hline
	3.5 & Ability to define a user-defined moment-rotation response representing the connection details & SP & D & 2.2 \\ \hline
	3.6 & Ability to quickly create a simple nonlinear building model & GC & D & 1.1 \\ \hline
	4 & \textbf{FEM} &  &  \\ \hline
	4.1 & Ability to specify OpenSees as FEM engine and to specify different analysis options & SP & M & 1.0 \\ \hline
	4.2 & Ability to provide own OpenSees Analysis script to OpenSees engine. & SP & D & 1.0 \\ \hline
	4.3 & Ability to provide own Python script and use OpenSeesPy engine. & SP & O & 1.2 \\ \hline
	4.4 & Ability to use alternative FEM engine. & SP & M & 2.0 \\ \hline
	5 & \textbf{UQ - Method} &  GC &  \\ \hline
	5.1 & \textbf{UQ - Forward Propogation Methods} & SP  &  \\ \hline
	5.1.1 & Ability to use basic  Monte Carlo and LHS methods & SP & M & 1.0 \\ \hline
	5.1.2 & Ability to use Importance Sampling  & SP & M & 2.0 \\ \hline
	5.1.3 & Ability to use Gaussian Process Regression & SP & M & 2.0 \\ \hline
	5.1.4 & Ability to use Own External UQ Engine & SP & M &  \\ \hline
	5.2 & \textbf{UQ - Reliability Methods} & UF &  &  \\ \hline
	5.2.1 & Ability to use First Order Reliability method & UF & M &  \\ \hline
	5.2.2 & Ability to use Second Order Reliability method & UF & M & \\ \hline
	5.2.2 & Ability to use Surrogate Based Reliability & UF & M & \\ \hline
	5.2.3 & Ability to use Own External Application to generate Results & UF & M &  \\ \hline
	5.3 & \textbf{UQ - Sensitivity Methods} & UF &  &  \\ \hline
	5.3.1 & Ability to obtain Global Sensitivity Sobol's indices & UF & M &  \\ \hline
    6 & \textbf{UQ – Random Variables} &  &  \\ \hline
    6.1 & Ability to Define Variables of certain types: & GC &  &  \\ 
    6.1.1 &  Normal & SP & M  & 1.0 \\ \hline
    6.1.2 &  Lognormal & SP & M & 1.0 \\ \hline
    6.1.3 & Uniform & SP & M & 1.0  \\ \hline
    6.1.4 & Beta & SP & M & 1.0 \\ \hline
    6.1.5 & Weibull &  SP & M  & 1.0 \\ \hline
    6.1.6 & Gumbel &  SP & M & 1.0  \\ \hline
    6.2 & User defined Distribution & SP & M &  \\ \hline
    6.3 & Define Correlation Matrix & SP & M &  \\ \hline
     7 & Tool to allow user to load and save user inputs & SP & M & 1.0 \\ \hline
    8 & \textbf{Engineering Demand Parameters} &  &  \\ \hline
    8.1 & Ability to Process own Output Parameters & UF & M & 1.1  \\ \hline
    8.2 & Add to Standard Earthquake a variable indicating analysis failure & UF & D &   \\ \hline
	\bottomrule 
\caption{Feature Requirements (M=Mandatory, D=Desirable, O=Optional, P=Possible Future)}             
  \label{tab:featureRequirements}                 
\end{longtable}

