\begin{longtable}{| p{.07\textwidth} | p{.65\textwidth} | p{.08\textwidth} | p{.08\textwidth} |  p{.08\textwidth} |}

  \caption{Requirements for EE-UQ}
  \label{tab:featureRequirements}    
     \\
   \hline
\rowcolor{lightgray}

      \# & Description & Source & Priority & Version \\ \hline
      E1 & \textbf{Ability to determine response of Building Subject to Earthquake hazard including formal treatment of randomness and uncertainty uncertainty} & GC & M  & 1.0  \\ \hline
 E1.1 & Ability of Practicing Engineers to use multiple coupled resources (applications, databases, viz tools) in engineering practice & GC & M 1.0 & \\ \hline
E1.2 & Ability to utilize resources beyond the desktop including HPC & GC & M & 1.0 \\ \hline
E1.3 & Tool should incorporate data from www & GC & M & 1.0 \\ \hline
E1.4 & Tool available for download from web & GC & M & 1.0 \\ \hline
E1.5 & Ability to benefit from programs that move research results into practice and obtain training & GC & M & \\ \hhline{=====}

      E2 & \textbf{Ability to select from different Input Motion Options} & SP & M & 1.0  \\ \hline
      E2.1 & Ability to select from Multiple input motions and view UQ due to all the discrete events & GC & M & 1.0  \\ \hline
      E2.2 & Ability to select from list of SimCenter motions & SP & M & 1.0 \\ \hline
      E2.3 & Ability to select from list of PEER motions & SP & D & 1.0 \\ \hline
      E2.4 & Ability to use OpenSHA and selection methods to generate motions & UF & D & 1.0 \\ \hline
      E2.5 & Ability to Utilize Own Application in Workflow & SP & M & 1.0 \\ \hline
      E2.6 & Ability to use Broadband & SP & D &  \\ \hline
      E2.7  & Ability to include Soil Structure Interaction Effects & GC & M & 1.1 \\  \hline
      E2.7.1  & 1D nonlinear site response with effective stress analysis & SP & M & 1.1  \\ \hline
      E2.7.2  & Nonlinear site response with bidirectional loading & SP & M & 1.2 \\  \hline
      E2.7.3  & Nonlinear site response with full stochastic characterization of soil layers & SP & M &  \\ \hline
      E2.7.4 & Nonlinear site response, bidirectional different input motions  & SP & M &  \\  \hline
      E2.7.5 & Ability to couple models from point of rupture through rock and soil into structure, which represents future of professional design practice & GC & M &  \\  \hline
      E2.7.5.1 & Interface using DRM method  & SP  & M &  \\  \hline
      E2.8 & Utilize PEER NGA www ground motion selection tool  & UF & D & 2.0 \\ \hline
      E2.9 & Ability to select from synthetic ground motions & SP & M & 1.0  \\
      E2.9.1 & per Vlachos, Papakonstantinou, Deodatis (2017) & SP & D & 1.1  \\ 
      E2.9.2 & per Dabaghi, Der Kiureghian (2017) & UF & D & 2.0 \\  \hhline{=====}
      
      
	E3 & \textbf{Ability to select different Building Model Generators} & GC & M & 1.0 \\ \hline
	E3.1 & Ability to quickly create a simple nonlinear building model for simple methods of seismic evaluation & GC 2.T13 & D & 1.1 \\ \hline
	E3.2 & Ability to use existing OpenSees model scripts & SP & M & 1.0 \\ \hline
	E3.3  & Ability to define building and use Expert System to generate FE mesh & SP & &  \\ \hline
	E3.3.1 & Expert system for Concrete Shear Walls & SP & M &  \\ \hline
	E3.3.2 & Expert system for Moment Frames & SP & M &  \\ \hline
	E3.3.3 & Expert system for  Braced Frames & SP & M &   \\ \hline
	E3.4 & Ability to define building and use Machine Learning applications to generate FE & GC &  &  \\ \hline
	E3.4.1 & Machine Learning for Concrete Shear Walls & SP & M &  \\ \hline
	E3.4.2 & Machine Learning for Moment Frames & SP & M &  \\ \hline
	E3.4.3 & Machine Learning for Braced Frames & SP & M &   \\ \hline
	E3.5 & Ability to specify connection details for member ends & UF & M & 2.2 \\ \hline
	E3.6 & Ability to define a user-defined moment-rotation response representing the connection details & UF & D & 2.2 \\  \hhline{=====}
	
	
	E4 & \textbf{Ability to select from different Nonlinear Analysis options} & GC & M & 1.0 \\ \hline
	E4.1 & Ability to specify OpenSees as FEM engine and to specify different analysis options & SP & M & 1.0 \\ \hline
	E4.2 & Ability to provide own OpenSees Analysis script to OpenSees engine. & SP & D & 1.0 \\ \hline
	E4.3 & Ability to provide own Python script and use OpenSeesPy engine. & SP & O & 1.2 \\ \hline
	E4.4 & Ability to use alternative FEM engine. & SP & M & 2.0 \\ \hhline{=====}

    E5 & \textbf{Engineering Demand Parameters} & SP & M &  \\ \hline
    E5.1 & Ability to specify standardized set of outputs & SP & M & 1.0  \\ \hline
    E5.2 & Ability to Process own Output Parameters & UF & M & 1.1  \\ \hline
    E5.3 & Add to Standard Earthquake a variable indicating analysis failure & UF & D &   \\ \hhline{=====}


U & \textbf{Ability to use various UQ Methods} & GC & M &  \\ \hline
U1 & \textbf{Forward Propogation Methods} & GC  & M & 1.0 \\ \hline
U1.1 & Ability to use basic Monte Carlo and LHS methods & SP & M & 1.0 \\ \hline
U1.2 & Ability to use Importance Sampling  & SP & M & 2.0 \\ \hline
U1.3 & Ability to use Gaussian Process Regression & SP & M & 2.0 \\ \hline
U1.4 & Ability to use Own External UQ Engine & SP & M &  \\ \hline
U2 & \textbf{Ability to use various Reliability Methods} & UF & M & 1.0 \\ \hline
U2.1 & Ability to use First Order Reliability method & UF & M &  \\ \hline
U2.2 & Ability to use Second Order Reliability method & UF & M & \\ \hline
U2.2 & Ability to use Surrogate Based Reliability & UF & M & \\ \hline
U2.3 & Ability to use Own External Application to generate Results & UF & M &  \\ \hline
U3 & \textbf{Ability to user various Sensitivity Methods} & UF & M & 1.0  \\ \hline
U3.1 & Ability to obtain Global Sensitivity Sobol's indices & UF & M &  \\ \hline
U4 & \textbf{Various Random Variable Probability Distributions} & SP & M & 1.0 \\ \hline
U4.1 & Ability to Define Variables of different types: & SP & M & 1.0  \\ \hline
U4.1.1 & Normal & SP & M  & 1.0 \\ \hline
U4.1.2 & Lognormal & SP & M & 1.0 \\ \hline
U4.1.3 & Uniform & SP & M & 1.0  \\ \hline
U4.1.4 & Beta & SP & M & 1.0 \\ \hline
U4.1.5 & Weibull &  SP & M  & 1.0 \\ \hline
U4.1.6 & Gumbel &  SP & M & 1.0  \\ \hline
U4.2 & User defined Distribution & SP & M &  \\ \hline
U4.3 & Define Correlation Matrix & SP & M &  \\ \hline
U4.4 & Random Fields & SP & M &  \\ \hhline{=====}



    EE & \textbf{Ability to obtain Educational material} &  &  & \\ \hline
    EE1 & Ability to use educational provisions to gain interdisclipinary education so as to gain expertise in earth sciences and physics, engineering mechanics, geotechnical engineering, and structural engineering in order to be qualified to perform these simulations & GC & D & \\ \hline
    EE2 & Documentation exists on tool usage & SP & M & 1.1  \\ \hline
    EE3 & Video Exists demonstrating usage & SP & M & 1.1  \\ \hline
    EE4 & Verification Examples Exist & SP & M & 1.1  \\ \hhline{=====}
    WE5 & Validation Examples Exist, validated against tests or other software & GC & M &  \\\hhline{=====}
    EM & \textbf{Misc.} &  & & \\ \hline
    EM1 & Tool to allow user to load and save user inputs & SP & M & 1.0 \\ \hline
    EM2 & Add to Standard Earthquake a variable indicating analysis failure & UF & D &   \\ \hline
    EM3 & Installer which installs application and all needed software & UF & M &   \\ \hline
      
  \bottomrule 
               
\end{longtable}

\noindent
KEY:\\
Source: GC=Needed for Grand Challenges, SP=Senior Personnel, UF=User Feedback \\
Need: M=Mandatory, D=Desirable, P=Possible Future \\
Version: Version number the basic requirement was met 
