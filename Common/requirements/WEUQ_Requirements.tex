\begin{longtable}{| p{.07\textwidth} | p{.65\textwidth} | p{.08\textwidth} | p{.08\textwidth} |  p{.08\textwidth} |}

\caption{Requirements for WE-UQ}
  \label{tab:featureRequirements}  
    \\
   \hline
\rowcolor{lightgray}
\# & Description & Source & Priority & Version \\ \hline

W1 & \textbf{Ability to determine response of Building Subject to Wind Loading including formal treatment of randomness and uncertainty uncertainty} & GC & M & 1.0  \\ \hline
W1.1 & Ability of Practicing Engineers to use multiple coupled resources (applications, databases, viz tools) in engineering practice & GC 5.IC4 & M 1.0 & \\ \hline
W1.2 & Ability to utilize resources beyond the desktop including HPC & GC 5.IC4 & M & 1.0 \\ \hline
W1.3 & Tool available for download from web & GC & M & 1.0 \\ \hline
W1.4 & Ability to obtain training and education with respect to interaction of structure and wind to ensure research is appropriately applied  & GC 5.IC3 & M & \\ \hline
W2 & \textbf{Multiple Wind Loading Options } & SP & M & 1.0 \\ \hline
W2.1 & Utilize Extensive wind tunnel datasets in industry and academia for wide range of building shapes & GC & M & 2.0 \\ \hline
W2.1.1 & Accommodate Range of Low Rise building shapes & SP & M & 2.0 \\ \hline
W2.1.1.1 & Flat Shaped Roof - TPU dataset & SP & M & 2.0 \\ \hline
W2.1.1.2 & Gable Shaped Roof - TPU dataset & SP & M & \\ \hline
W2.1.1.3 & Hipped Shaped Roof - TPU dataset & SP & M & \\ \hline
W2.1.2 & Accommodate Range of High Rise building shapes & SP & M & 1.0 \\ \hline
W2.1.2.1 & Interface with Vortex Winds DEDM-HRP Web service & SP & M & 1.0 \\ \hline
W2.1.3 & Accommodate Data from Wind Tunnel Experiment & SP & M & 2.0 \\ \hline
W2.1.3.1 & Cuboid - User Provided Wind Tunnel Experiment Data  & SP & M & 2.0 \\ \hline
W2.2 & Computational Fluid Dynamics tool for utilizing open source CFD software for practitioners & GC 6.4 & M & 1.0 \\ \hline
W2.2.1 & Simple CFD model generation and turbulence modeling & GC 6.4 & M & 2.0 \\ \hline
W2.2.2 & Uncoupled OpenFOAM CFD model with nonlinear FEM code for building response & SP & M & 1.0 \\ \hline
W2.2.3 & Coupled OpenFOAM CFD model with nonlinear FEM code for building response & SP & M &  \\ \hline
W2.3 & Quantification of Effects of Wind Borne Debris & GC 6.4 & D & \\ \hline
W2.4 & Application to utilize GIS and online to account for wind speed given local terrain, topography and nearby buildings & GC & D & \\ \hline
W2.5 & Ability to utilize synthetic wind loading algorithms & SP & M & 1.0  \\ \hline
W2.5.1 & per Wittig and Sinha & SP & D & 1.1  \\ \hline
W3 & \textbf{Building Model Generation} & GC & M & 2.0 \\ \hline
W3.1 & Ability to quickly create a simple nonlinear building model & GC & D & 1.1 \\ \hline
W3.2  & Ability to define building and use Expert System to generate FE mesh & SP & &  \\ \hline
	3.2.1 & Expert system for Concrete Shear Walls & SP & M &  \\ \hline
	3.2.2 & Expert system for Moment Frames & SP & M &  \\ \hline
	3.2.3 & Expert system for  Braced Frames & SP & M &   \\ \hline
W3.3 & Ability to define building and use Machine Learning applications to generate FE & GC &  &  \\ \hline
	W3.3.1 & Machine Learning for Concrete Shear Walls & SP & M &  \\ \hline
	W3.3.2 & Machine Learning for Moment Frames & SP & M &  \\ \hline
	W3.3.3 & Machine Learning for Braced Frames & SP & M &   \\ \hline
	W3.4 & Ability to specify connection details for member ends & SP & M & 2.2 \\ \hline
	W3.5 & Ability to define a user-defined moment-rotation response representing the connection details & SP & D & 2.2 \\ \hline
	W4 & \textbf{Perform Nonlinear Structural Analysis} & GC & M & 1.0 \\ \hline
W4.1 & Ability to use utilize existing nonlinear analysis software used in earthquake engineering & GC & M & 1.0 \\ \hline
W4.1.1 & Utilize open source OpenSees software & SP & M & 1.0 \\ \hline
W4.2.1 & Ability to provide own OpenSees Analysis script to OpenSees engine. & SP & D & 1.0 \\ \hline
W4.3.1 & Ability to provide own Python script and use OpenSeesPy engine. & SP & O & 1.2 \\ \hline
W4.2 & Ability to use alternative FEM engine & SP & M & 2.0 \\ \hline
U & \textbf{Ability to use various UQ Methods} & GC & M &  \\ \hline
U1 & \textbf{Forward Propogation Methods} & GC  & M & 1.0 \\ \hline
U1.1 & Ability to use basic Monte Carlo and LHS methods & SP & M & 1.0 \\ \hline
U1.2 & Ability to use Importance Sampling  & SP & M & 2.0 \\ \hline
U1.3 & Ability to use Gaussian Process Regression & SP & M & 2.0 \\ \hline
U1.4 & Ability to use Own External UQ Engine & SP & M &  \\ \hline
U2 & \textbf{Ability to use various Reliability Methods} & UF & M & 1.0 \\ \hline
U2.1 & Ability to use First Order Reliability method & UF & M &  \\ \hline
U2.2 & Ability to use Second Order Reliability method & UF & M & \\ \hline
U2.2 & Ability to use Surrogate Based Reliability & UF & M & \\ \hline
U2.3 & Ability to use Own External Application to generate Results & UF & M &  \\ \hline
U3 & \textbf{Ability to user various Sensitivity Methods} & UF & M & 1.0  \\ \hline
U3.1 & Ability to obtain Global Sensitivity Sobol's indices & UF & M &  \\ \hline
U4 & \textbf{Various Random Variable Probability Distributions} & SP & M & 1.0 \\ \hline
U4.1 & Ability to Define Variables of different types: & SP & M & 1.0  \\ \hline
U4.1.1 & Normal & SP & M  & 1.0 \\ \hline
U4.1.2 & Lognormal & SP & M & 1.0 \\ \hline
U4.1.3 & Uniform & SP & M & 1.0  \\ \hline
U4.1.4 & Beta & SP & M & 1.0 \\ \hline
U4.1.5 & Weibull &  SP & M  & 1.0 \\ \hline
U4.1.6 & Gumbel &  SP & M & 1.0  \\ \hline
U4.2 & User defined Distribution & SP & M &  \\ \hline
U4.3 & Define Correlation Matrix & SP & M &  \\ \hline
U4.4 & Random Fields & SP & M &  \\ \hhline{=====}



     W7 & Tool to allow user to load and save user inputs & SP & M & 1.0 \\ \hline
    W8 & \textbf{Application Outputs} &  &  \\ \hline
    W8.1 & Ability to see pressure distribution on building & GC & M &   \\ \hline
    W8.2 & Ability to obtain basic building responses & SP & M &   \\ \hline
    W8.3 & Ability to Process own Output Parameters & UF & M & 1.1  \\ \hline
    W9 & \textbf{Documentation} &  &  \\ \hline
    W9.1 & Documentation exists on tool usage & SP & M & 1.1  \\ \hline
    W9.2 & Video Exists demonstrating usage & SP & M & 1.1  \\ \hline
    W9.3 & Verification Examples Exist & SP & M & 1.1  \\ \hline
    W10 & \textbf{Misc.} &  &  \\ \hline
    W10.1 & Installer  which installs application and all needed software & UF & M &   \\ \hline
	\bottomrule 
               
\end{longtable}

\noindent
KEY:\\
Source: GC=Needed for Grand Challenges, SP=Senior Personnel, UF=User Feedback \\
Need: M=Mandatory, D=Desirable, P=Possible Future \\
Version: Version number the basic requirement was met 


