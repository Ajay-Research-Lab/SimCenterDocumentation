%% Chapter: Introduction

The Turbulent Inflow Tool (TInF) is designed to collect all required properties and parameters needed for various turbulent inflow models in OpenFOAM, and to augment an existing wind-around-a-building model by adding the necessary sections to respective parameter definition files.

The generic workflow involved is as follows.
\begin{enumerate}
\item
   Build your OpenFOAM model as you would without using a turbulent inflow model.  Use a generic patch with a suitable name for you will need to identify that patch by its name inside TInF.
   
\item
   Run TInF following, identify your model folder, adjust the parameters as desired, and export to your model definition.
   Consult Chapter~\ref{sec:TInF-usage} for details on those steps.
   
\item
    Run OpenFOAM using the updated model definition.
    
\end{enumerate}

The tool also provides a Save to file and Open from file functionality that will allow you to define and share complex sets of settings and parameters for the supported turbulent inflow models and, such, efficiently and reliably apply the same parameters to several different models.

%% leave this as overview
\noindent
The following Chapter~\ref{sec:TInF-installation} explains the installation of the tool and how to update your local OpenFOAM copy to implement the supported turbulent inflow models.
Chapter~\ref{sec:TInF-usage} will walk you through the steps required to add a turbulent inflow condition to your model.
Chapter~\ref{sec:TInF-theory} provides a detailed theoretical background on the provided turbulence models.