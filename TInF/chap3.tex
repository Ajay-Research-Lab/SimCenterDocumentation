\label{sec:TInF-installation}

\section{Installing the Turbulent Inflow Tool}

Download the installation package for your operation system from (a single line)
\begin{verbatim}
https://www.designsafe-ci.org/data/browser/public/designsafe.storage.community/
            /SimCenter/Software/TurbulantInflowTool
\end{verbatim}
 SimCenter is providing packages for Windows~8/10 (64 bit version only) and MacOS.  
The installer will place the executable on your system.  On Windows systems, a shortcut will be added to your start menu.
On MacOS, the application is placed in your Applications folder.
\bigskip

For Linux systems, you will need to clone the source from 
\begin{verbatim}
https://github.com/NHERI-SimCenter/TurbulentInflowTool
\end{verbatim}
and compile it yourself performing the following steps:
\begin{quote}
\begin{verbatim}
$ git clone https://github.com/NHERI-SimCenter/TurbulentInflowTool
$ git clone https://github.com/NHERI-SimCenter/SimCenterCommon
$ cd TurbulentInflowTool
$ qmake TurbulentInflowTool.pro
$ make
$ sudo make install
\end{verbatim}
\end{quote}

\section{Compiling the Source Code in OpenFOAM}

Download the source code of the turbulent velocity boundary conditions from
\begin{verbatim}
https://github.com/NHERI-SimCenter/TurbulentInflowTool/tree/master/openFOAM_code
\end{verbatim}

\noindent  The source code files are contained in a directory named \textcolor{blue}{inflowTurbulence}. Note that the code is provided for OpenFOAM version 6 and 7, and is slightly different in this two versions. Please choose the correct package to download according the version of OpenFOAM you have installed on your computer.

Create a project directory named \textcolor{blue}{run} within the \textcolor{blue}{\$HOME/OpenFOAM} directory named \textless{USER}\textgreater{-6} (e.g. Jack-6 for user Jack and OpenFOAM version 6) by typing the following script in a terminal prompt:

\begin{quote}
\begin{verbatim}
$ mkdir -p $FOAM_RUN
\end{verbatim}
\end{quote}

Copy or move the \textcolor{blue}{inflowTurbulence} directory which has been downloaded earlier and all the files in it to the \textcolor{blue}{\$HOME/OpenFOAM} directory. Go the relocated \textcolor{blue}{inflowTurbulence} directory by typing:

\begin{quote}
\begin{verbatim}
$ cd $FOAM_RUN/inflowTurbulence
\end{verbatim}
\end{quote}

Compile the codes in the \textcolor{blue}{inflowTurbulence} directory by typing the following in the terminal prompt:

\begin{quote}
\begin{verbatim}
$ wmake
\end{verbatim}
\end{quote}

After the compilation is successfully complete, a library file named \textcolor{blue}{libturbulentInflow.so} will be generated in the directory \textcolor{blue}{\$FOAM\_USER\_LIBBIN}.
